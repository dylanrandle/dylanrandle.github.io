%%%%%%%%%%%%%%%%%%%%%%%%%%%%%%%%%%%%%%%%%
% Medium Length Professional CV
% LaTeX Template
% Version 2.0 (8/5/13)
%
% This template has been downloaded from:
% http://www.LaTeXTemplates.com
%
% Original author:
% Trey Hunner (http://www.treyhunner.com/)
%
% Important note:
% This template requires the resume.cls file to be in the same directory as the
% .tex file. The resume.cls file provides the resume style used for structuring the
% document.
%
%%%%%%%%%%%%%%%%%%%%%%%%%%%%%%%%%%%%%%%%%

%----------------------------------------------------------------------------------------
%	PACKAGES AND OTHER DOCUMENT CONFIGURATIONS
%----------------------------------------------------------------------------------------

\documentclass{resume} % Use the custom resume.cls style

\usepackage[left=0.75in,top=0.6in,right=0.75in,bottom=0.6in]{geometry} % Document margins
\newcommand{\tab}[1]{\hspace{.2667\textwidth}\rlap{#1}}
\newcommand{\itab}[1]{\hspace{0em}\rlap{#1}}
\name{Dylan Labatt Randle} % Your name
% \address{B-1 \\ II , U.P. 208016} % Your address
%\address{123 Pleasant Lane \\ City, State 12345} % Your secondary addess (optional)
\address{+1-647-641-1994 \\ \url{dylanrandle@g.harvard.edu} \\ \url{dylanrandle.github.io}} % Your phone number and email

\begin{document}

%----------------------------------------------------------------------------------------
%	EDUCATION SECTION
%----------------------------------------------------------------------------------------

\begin{rSection}{Education}

{\bf Harvard University} \hfill { Cambridge, MA}
\\ {\em M.S. in Data Science} \hfill {\em Expected May 2020}
\\ Relevant coursework: Advanced Data Science, Stochastic Methods for Optimization, Modeling and Inference, Systems Development for Computational Science, Computational Science Seminar
%Minor in Linguistics \smallskip \\
%Member of Eta Kappa Nu \\
%Member of Upsilon Pi Epsilon \\

{\bf University of California, Berkeley} \hfill {Berkeley, CA}
\\ {\em B.S. in Industrial Engineering \& Operations Research, High Honors} \hfill {\em May 2016}
\\ Relevant coursework: Statistics and Machine Learning, Probability, Forecasting, Mathematical Programming, Nonlinear and Discrete Optimization, Stochastic Processes


\end{rSection}

%----------------------------------------------------------------------------------------
%	WORK EXPERIENCE SECTION
%----------------------------------------------------------------------------------------

\begin{rSection}{Relevant Experience}

\begin{rSubsection}{Institute for Applied Computational Science, Harvard University}{Cambridge, MA}{Research Assistant}{Nov 2018 - Present}
\item Researching physics-aware machine learning methods for turbulence modeling, supervised by Pavlos Protopapas and David Sondak
\end{rSubsection}

%------------------------------------------------

\begin{rSubsection}{Hubdoc}{Toronto, Canada}{Data Scientist \& Machine Learning Engineer}{ Feb 2017 - July 2018}
\item Developed and deployed deep learning system (LSTM \& CNN) for information extraction and text classification from financial documents. Saved \$1MM+ per year by decreasing labor costs and service times by orders of magnitude. Used Python, Tensorflow, Keras, AWS, PostgresSQL, Ansible
\item Built data visualizations for company intranet. Presented results and recommendations to management team; delivered introductory machine learning lecture to audience of 60+ people
\end{rSubsection}

%------------------------------------------------

\end{rSection}

% RELEVANT PROJECTS
%----------------------------------------------------------------------------------------

\begin{rSection}{Relevant Projects}

\begin{rSubsection}{Twitter Troll Detection: \url{https://dylanrandle.github.io/troll_classification}}{}{}{}
\item Achieved 96\% accuracy in classifying tweets as trolls, using a dataset of Twitter handles from the Internet Research Agency, an organization indicted by prosecutors for meddling in the 2016 U.S. presidential election
\end{rSubsection}

%------------------------------------------------

\begin{rSubsection}{Automatic Differentiation: \url{https://github.com/dylanrandle/autograd }}{}{}{}
\item Built a Python package implementing automatic differentiation (forward and reverse mode) in NumPy. Includes gradient descent and Adam optimizers, with extensive documentation.
\end{rSubsection}

\end{rSection}

% LEADERSHIP
%----------------------------------------------------------------------------------------

\begin{rSection}{LEADERSHIP EXPERIENCE}

\begin{rSubsection}{Taylor Statten Camps}{Algonquin Park, Canada}{Long Trip Counselor}{Summer 2015/2016}
\item Led 50-day and 36-day wilderness canoe trips through Northern Ontario and Minnesota. Planned route and food drops, navigated 2500km+ of rugged wilderness, and ensured safety of numerous groups of 7 teenage boys
\end{rSubsection}

\end{rSection}

%----------------------------------------------------------------------------------------
%	TECHNICAL STRENGTHS SECTION
%----------------------------------------------------------------------------------------

\begin{rSection}{Technical Skills}

\begin{tabular}{ @{} >{\bfseries}l @{\hspace{6ex}} l }
Proficient &  Python (numpy, pandas, scikit-learn, pytorch, tensorflow, pymc3), SQL, Git \\
Familiar & Javascript, C++, MATLAB, Latex \\
\end{tabular}

\end{rSection}

\end{document}
