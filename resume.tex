%%%%%%%%%%%%%%%%%%%%%%%%%%%%%%%%%%%%%%%%%
% Medium Length Professional CV
% LaTeX Template
% Version 2.0 (8/5/13)
%
% This template has been downloaded from:
% http://www.LaTeXTemplates.com
%
% Original author:
% Trey Hunner (http://www.treyhunner.com/)
%
% Important note:
% This template requires the resume.cls file to be in the same directory as the
% .tex file. The resume.cls file provides the resume style used for structuring the
% document.
%
%%%%%%%%%%%%%%%%%%%%%%%%%%%%%%%%%%%%%%%%%

%----------------------------------------------------------------------------------------
%	PACKAGES AND OTHER DOCUMENT CONFIGURATIONS
%----------------------------------------------------------------------------------------

\documentclass{resume} % Use the custom resume.cls style

\usepackage[left=0.75in,top=0.6in,right=0.75in,bottom=0.6in]{geometry} % Document margins
\newcommand{\tab}[1]{\hspace{.2667\textwidth}\rlap{#1}}
\newcommand{\itab}[1]{\hspace{0em}\rlap{#1}}
\name{Dylan Labatt Randle} % Your name
% \address{B-1 \\ II , U.P. 208016} % Your address
%\address{123 Pleasant Lane \\ City, State 12345} % Your secondary addess (optional)
\address{+1-647-641-1994 \\ \url{dylanrandle@g.harvard.edu} \\ \url{dylanrandle.github.io}} % Your phone number and email

\begin{document}

%----------------------------------------------------------------------------------------
%	EDUCATION SECTION
%----------------------------------------------------------------------------------------

\begin{rSection}{Education}

\begin{rSubsection}{Harvard University}{Cambridge, MA}{M.S. in Data Science}{Expected May 2020}
\item Relevant coursework: Advanced Data Science, Stochastic Optimization, Bayesian Inference, High Performance Parallel Computing
\end{rSubsection}

% Minor in Linguistics \smallskip \\
% Member of Eta Kappa Nu \\
% Member of Upsilon Pi Epsilon \\

\begin{rSubsection}{University of California, Berkeley}{Berkeley, CA}{B.S. in Industrial Engineering \& Operations Research, GPA:3.9/4.0}{May 2016}
\item Honors: High Honors at Graduation, Phi Beta Kappa, Tau Beta Pi
\item Relevant coursework: Statistics, Machine Learning, Optimization, Simulation, Decision Theory
\end{rSubsection}


\end{rSection}

%----------------------------------------------------------------------------------------
%	WORK EXPERIENCE SECTION
%----------------------------------------------------------------------------------------

\begin{rSection}{Relevant Experience}

\begin{rSubsection}{Harvard University}{Cambridge, MA}{Research Assistant}{Nov 2018 - Present}
\item Researching physics-aware neural networks for solving partial differential equations. Supervised by Pavlos Protopapas and David Sondak.
\end{rSubsection}

%------------------------------------------------

\begin{rSubsection}{Hubdoc}{Toronto, Canada}{Data Scientist}{Feb 2017 - July 2018}
\item Developed and deployed deep learning system using LSTMs \& CNNs for information extraction and text classification from financial documents. Greatly reduced cost (\$1-3MM/year) and increased speed (14,000x faster for 80\% of documents) of results. Used Python, Keras, Postgres, Ansible, AWS.
\item Conducted analyses (work prioritization, labor allocation, anomaly detection) as needed. Built data visualizations for company intranet. Presented results and recommendations to management team. Delivered introductory machine learning lecture to audience of 60+ people.
\end{rSubsection}

%------------------------------------------------

\begin{rSubsection}{BMO Capital Markets}{Toronto, Canada}{Financial Products Analyst}{May 2014 - Aug 2014}
\item Conducted analyses of various debt products (swaps, swaptions, ABS, MBS). Wrote algorithm in C\# to analyze relationship between delta-hedging frequency and returns for Canadian swaptions; found possible trading opportunities.
\end{rSubsection}

\end{rSection}

% RELEVANT PROJECTS
%----------------------------------------------------------------------------------------

\begin{rSection}{Relevant Projects}

\begin{rSubsection}{Twitter Troll Detection: \url{https://dylanrandle.github.io/troll_classification}}{}{}{}
\item Achieved 96\% accuracy in classifying tweets as trolls, using a dataset of Twitter handles indicted for meddling in the 2016 U.S. presidential election.
\end{rSubsection}

%------------------------------------------------

\begin{rSubsection}{Automatic Differentiation: \url{https://github.com/dylanrandle/autograd }}{}{}{}
\item Built a Python package implementing automatic differentiation (forward and reverse mode). Used to implement gradient descent and Adam optimizers, with extensive documentation.
\end{rSubsection}

\end{rSection}

%----------------------------------------------------------------------------------------
%	TECHNICAL STRENGTHS SECTION
%----------------------------------------------------------------------------------------

\begin{rSection}{Technical Skills}

\begin{tabular}{ @{} >{\bfseries}l @{\hspace{6ex}} l }
Proficient &  Python (numpy, pandas, scikit-learn, pytorch, keras, pymc3), SQL, Git \\
Familiar & Javascript, C++, MATLAB, Latex \\
\end{tabular}

\end{rSection}

\end{document}
