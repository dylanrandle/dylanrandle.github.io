%%%%%%%%%%%%%%%%%%%%%%%%%%%%%%%%%%%%%%%%%
% Medium Length Professional CV
% LaTeX Template
% Version 2.0 (8/5/13)
%
% This template has been downloaded from:
% http://www.LaTeXTemplates.com
%
% Original author:
% Trey Hunner (http://www.treyhunner.com/)
%
% Important note:
% This template requires the resume.cls file to be in the same directory as the
% .tex file. The resume.cls file provides the resume style used for structuring the
% document.
%
%%%%%%%%%%%%%%%%%%%%%%%%%%%%%%%%%%%%%%%%%

%----------------------------------------------------------------------------------------
%	PACKAGES AND OTHER DOCUMENT CONFIGURATIONS
%----------------------------------------------------------------------------------------

\documentclass{resume} % Use the custom resume.cls style

\usepackage[left=0.75in,top=0.6in,right=0.75in,bottom=0.6in]{geometry} % Document margins
\newcommand{\tab}[1]{\hspace{.2667\textwidth}\rlap{#1}}
\newcommand{\itab}[1]{\hspace{0em}\rlap{#1}}
\name{Dylan Labatt Randle} % Your name
% \address{B-1 \\ II , U.P. 208016} % Your address
%\address{123 Pleasant Lane \\ City, State 12345} % Your secondary addess (optional)
\address{\url{dylanrandle@g.harvard.edu} \\ \url{dylanrandle.github.io}} % Your phone number and email

\begin{document}

%----------------------------------------------------------------------------------------
%	EDUCATION SECTION
%----------------------------------------------------------------------------------------

\begin{rSection}{Education}

{\bf Harvard University, School of Engineering \& Applied Sciences} \hfill { Cambridge, MA} 
\\ {\em M.S. in Data Science} \hfill {\em May 2020 (Expected)}
\\ Scholarship in Applied Computation
%Minor in Linguistics \smallskip \\
%Member of Eta Kappa Nu \\
%Member of Upsilon Pi Epsilon \\

{\bf University of California at Berkeley, College of Engineering} \hfill {Berkeley, CA} 
\\ {\em B.S. in Industrial Engineering \& Operations Research} \hfill {\em May 2016}
\\ High Honors, Phi Beta Kappa, Tau Beta Pi


\end{rSection}

%----------------------------------------------------------------------------------------
%	TECHNICAL STRENGTHS SECTION
%----------------------------------------------------------------------------------------

\begin{rSection}{Technical Skills}

\begin{tabular}{ @{} >{\bfseries}l @{\hspace{6ex}} l }
Expert & Python (Numpy, Pandas, Scikit-Learn, Pytorch, Tensorflow) \\
Proficient & AWS, Apache Spark, SQL, Git, Jupyter, Latex \\
Familiar & Javascript, Matlab, C \\
\end{tabular}

\end{rSection}

%----------------------------------------------------------------------------------------
%	WORK EXPERIENCE SECTION
%----------------------------------------------------------------------------------------

\begin{rSection}{Work Experience}

\begin{rSubsection}{Amazon Robotics}{North Reading, MA}{Data Science Intern}{Jun 2019 - Aug 2019}
\item 
    Developed machine learning package for proprietary internal use cases. 
    Built automated and scalable data pipeline for big data querying, cleaning, and loading ($\sim 1 \times 10^{12}$ rows). 
    Implemented API for feature selection, model training, hyperparameter tuning, and testing. 
    Included interpretable visualizations (PDP, SHAP) for model explanations.

\item 
    Greatly increased speed and reduced complexity of model development. 
    Wrote documentation and published code to internal repositories. 
    % Tools: Python, Spark, AWS EMR
\end{rSubsection}

%------------------------------------------------

\begin{rSubsection}{Harvard University}{Cambridge, MA}{Graduate Researcher \& Teaching Fellow}{Jan 2019 -- Present}
\item 
    Developed method for training unsupervised generative adversarial networks to solve differential equations.
    Invented grid sampling procedure leading to improved convergence. Paper in progress.
\item 
    Applied decision sets and explainable boosting machines to reinforcement learning to learn interpretable policies targeted at healthcare applications. Paper in progress.
\item 
    Prepared lecture materials on boosting, neural networks, gradient descent, backpropagation, and regularization.
    Explanations and visualizations praised by students for their clarity and simplicity.
\end{rSubsection}

%------------------------------------------------

\begin{rSubsection}{Hubdoc (acquired by Xero)}{Toronto, Canada}{Data Scientist}{Jan 2017 - Jul 2018}
\item 
    First data scientist hired. 
    Grew team threefold while creating highly valuable ``text extraction" product, a crucial piece driving the Xero acquisition.
    
\item 
    Developed production deep learning system (LSTMs \& CNNs) for entity extraction and text classification of financial documents.
    Built scalable, asynchronous pipeline for serving predictions.
    Deployed fault-tolerant system to main web application, with live monitoring and alerting.
    % Tools: Python, Tensorflow, AWS
    
\item 
    Regularly presented results and recommendations to C-suite.
    Pitched machine learning strategy to investors.
    Delivered lectures to audiences of 40-60 people.
    Built data visualizations for company intranet. 
\end{rSubsection}

%------------------------------------------------

\end{rSection}

% RELEVANT PROJECTS
%----------------------------------------------------------------------------------------

\begin{rSection}{Relevant Projects}

Please see my website for all of my available work: \url{dylanrandle.github.io}

% \begin{rSubsection}{Twitter Troll Detection: \url{https://dylanrandle.github.io/troll_classification}}{}{}{}
% \item Achieved 96\% accuracy in classifying tweets as trolls, using a dataset of Twitter handles from the Internet Research Agency, an organization indicted by prosecutors for meddling in the 2016 U.S. presidential election
% \end{rSubsection}

% %------------------------------------------------

% \begin{rSubsection}{Automatic Differentiation: \url{https://github.com/dylanrandle/autograd }}{}{}{}
% \item Built a Python package implementing automatic differentiation (forward and reverse mode) in NumPy. Includes gradient descent and Adam optimizers, with extensive documentation.
% \end{rSubsection}

\end{rSection}

%----------------------------------------------------------------------------------------
% LEADERSHIP
%----------------------------------------------------------------------------------------

% \begin{rSection}{LEADERSHIP EXPERIENCE}

% \begin{rSubsection}{Taylor Statten Camps}{Algonquin Park, Canada}{Long Trip Counselor}{Summer 2015/2016}
% \item Led 50-day and 36-day wilderness canoe trips through Northern Ontario and Minnesota. Planned route and food drops, navigated 2500km+ of rugged wilderness, and ensured safety of numerous groups of 7 teenage boys
% \end{rSubsection}

% \end{rSection}

\end{document}

